%%%%%%%%%%%%%%%%%%%%%%%%%%%%%%%%%%%%%%%%%
% "ModernCV" CV and Cover Letter
% LaTeX Template
% Version 1.11 (19/6/14)
%
% This template has been downloaded from:
% http://www.LaTeXTemplates.com
%
% Original author:
% Xavier Danaux (xdanaux@gmail.com)
%
% License:
% CC BY-NC-SA 3.0 (http://creativecommons.org/licenses/by-nc-sa/3.0/)
%
% Important note:
% This template requires the moderncv.cls and .sty files to be in the same 
% directory as this .tex file. These files provide the resume style and themes 
% used for structuring the document.
%
%%%%%%%%%%%%%%%%%%%%%%%%%%%%%%%%%%%%%%%%%

%----------------------------------------------------------------------------------------
%	PACKAGES AND OTHER DOCUMENT CONFIGURATIONS
%----------------------------------------------------------------------------------------


\documentclass[10pt,letterpaper]{moderncv} % Font sizes: 10, 11, or 12; paper sizes: a4paper, letterpaper, a5paper, legalpaper, executivepaper or landscape; font families: sans or roman


\moderncvstyle{classic} % CV theme - options include: 'casual' (default), 'classic', 'oldstyle' and 'banking'
\moderncvcolor{orange} % CV color - options include: 'blue' (default), 'orange', 'green', 'red', 'purple', 'grey' and 'black'

\usepackage[scale=0.8]{geometry} % Reduce document margins
%\setlength{\hintscolumnwidth}{3cm} % Uncomment to change the width of the dates column
\setlength{\makecvtitlenamewidth}{40cm} % For the 'classic' style, uncomment to adjust the width of the space allocated to your name


\usepackage{hyperref}
\hypersetup{
    colorlinks = true,
    linkcolor = red
}

%----------------------------------------------------------------------------------------
%	NAME AND CONTACT INFORMATION SECTION
%----------------------------------------------------------------------------------------

\firstname{Maria [Masha]} % Your first name
\familyname{Okounkova} % Your last name

%\photo[4in][0.4pt]{me_lecture} % The first bracket is the picture height, the second is the thickness of the frame around the picture (0pt for no frame)

% All information in this block is optional, comment out any lines you don't need
%\title{Curriculum Vitae}
\address{Flatiron Institute, 162 5th Ave}{New York, NY, 10010}
\email{mokounkova@flatironinstitute.org}
\homepage{https://mariaokounkova.github.io/}


%----------------------------------------------------------------------------------------

\begin{document}

\makecvtitle % Print the CV title

I am a Flatiron Research Fellow at the Center for Computational Astrophysics at Simons Foundation Flatiron Institute in New York City. My research is in numerical relativity, and I am primarily interested in using numerical relativity to test general relativity through gravitational wave observations. I am a member of the \href{https://www.black-holes.org/}{Simulating Extreme Spacetimes (SXS)} collaboration and the \href{https://www.ligo.org/}{LIGO Scientific Collaboration (LSC)}.  


\section{Scientific Interests}
%
\cvitem{}{Numerical relativity, binary black holes, gravitational waves, theories of gravity beyond general relativity, testing general relativity with gravitational wave observations, black hole quasi-normal modes, binary black hole spacetime non-linearities, black hole shadows, code development for numerical relativity}


%----------------------------------------------------------------------------------------
%	POSITIONS
%----------------------------------------------------------------------------------------
\section{Academic positions}

\cventry{Aug 2019 - present}{Flatiron Institute Center for Computational Astrophysics (CCA)}{}{\textit{Flatiron Research Fellow}}{Member of Gravitational Waves and Compact Objects groups}{}{}

%----------------------------------------------------------------------------------------
%	EDUCATION
%----------------------------------------------------------------------------------------

\section{Education}

\cventry{2014 - 2019}{California Institute of Technology (Caltech)}{}{}{PhD in physics}{advised by Saul Teukolsky}{}

\cventry{2010 - 2014}{Princeton University}{}{}{B.A. in physics, certificate in applications of computing}{\textit{magna cum laude}}{}





%----------------------------------------------------------------------------------------
%	Publications
%----------------------------------------------------------------------------------------


\section{Selected Publications}

\cvitem{[11]}{\textbf{Maria Okounkova}, Will Farr, Maximilliano Isi, Leo C. Stein. \textit{Constraining gravitational wave amplitude birefringence and Chern-Simons gravity with GWTC-2}. \href{https://arxiv.org/abs/2101.11153}{arXiv:2101.11153} Submitted to Phys. Rev. D., Jan 2021}

\cvitem{[10]}{\textbf{Maria Okounkova}. \textit{Revisiting non-linearity in binary black hole mergers}. 
\href{https://arxiv.org/abs/2004.00671}{arXiv:2004.00671}
Submitted to Phys. Rev. D., Apr 2020 }


\cvitem{[9]}{\textbf{Maria Okounkova}. \textit{Numerical relativity simulation of GW150914 in Einstein dilaton Gauss-Bonnet gravity}. 
\href{https://journals.aps.org/prd/abstract/10.1103/PhysRevD.102.084046}{Phys. Rev. D 102:084046}, Oct 2020}


\cvitem{[8]}{\textbf{Maria Okounkova},  Leo C. Stein, Jordan Moxon, Mark A. Scheel,  and  Saul  A.  Teukolsky. \textit{Numerical relativity simulation of GW150914 beyond general relativity}.  \href{https://journals.aps.org/prd/abstract/10.1103/PhysRevD.101.104016}{Phys. Rev. D 101:104016}, May 2020}


\cvitem{[7]}{\textbf{Maria Okounkova}. \textit{Stability of rotating black holes in Einstein dilaton Gauss-Bonnet gravity}.  
\href{https://journals.aps.org/prd/abstract/10.1103/PhysRevD.100.124054}{Phys. Rev. D 100:124054}, Dec 2019}


\cvitem{[6]}{\textbf{Maria Okounkova},  Leo C. Stein, Mark A. Scheel,  and  Saul  A.  Teukolsky. \textit{Numerical binary black hole collisions in dynamical Chern-Simons gravity}.  
\href{https://journals.aps.org/prd/abstract/10.1103/PhysRevD.100.104026}{Phys. Rev. D 100:104026}, Nov 2019}

\cvitem{[5]}{Michael Boyle et al. (inc \textbf{Maria Okounkova}), \textit{The SXS Collaboration catalog of binary black hole simulations} \href{https://iopscience.iop.org/article/10.1088/1361-6382/ab34e2}{Class.  Quant.  Grav.}, April 2019}

\cvitem{[4]}{\textbf{Maria Okounkova},  Mark A. Scheel,  and  Saul  A.  Teukolsky. \textit{Evolving Metric Perturbations in dynamical Chern-Simons Gravity}.  
\href{https://journals.aps.org/prd/abstract/10.1103/PhysRevD.99.044019}{Phys. Rev. D 99:044019}, Feb 2019}

\cvitem{[3]}{\textbf{Maria Okounkova},  Mark A. Scheel,  and  Saul  A.  Teukolsky. \textit{Numerical black hole initial data and shadows in dynamical Chern-Simons gravity}. \href{http://iopscience.iop.org/article/10.1088/1361-6382/aafcdf}{Class.  Quant.  Grav.}, Feb 2019}
    


\cvitem{[2]}{Swetha Bhagwat, \textbf{Maria Okounkova},  Stefan  W.  Ballmer,  Duncan  A. Brown,  Matthew  Giesler,  Mark  A.  Scheel,  and  Saul  A.  Teukolsky. \textit{On choosing the start time of binary black hole ringdowns}. \href{https://link.aps.org/doi/10.1103/PhysRevD.97.104065}{Phys.  Rev.  D 97:104065}, May 2018.} 
    

\cvitem{[1]}{\textbf{Maria Okounkova}, Leo C. Stein, Mark A. Scheel, and Daniel A. Hemberger. \textit{Numerical binary black hole mergers in dynamical Chern-Simons gravity: Scalar field}. \href{https://link.aps.org/doi/10.1103/PhysRevD.96.044020}{Phys. Rev. D 96:044020}, Aug 2017.} 



\section{Upcoming Publications}

\cvitem{[2]}{\textbf{Maria Okounkova}, Francois Hebert, Katerina Chatziioannou, Jordan Moxon, Leo Stein, Saul Teukolsky \textit{Connecting the strong field dynamics of binary black hole mergers to gravitational waveforms at infinity using ray-tracing}. In prep, to be submitted to Phys. Rev. D.}


\cvitem{[1]}{\textbf{Maria Okounkova}, Maximilliano Isi, Katerina Chatziioannou, Will Farr. \textit{Searching for binary black hole mergers beyond general relativity}. In prep, to be submitted to Phys. Rev. D.}






%----------------------------------------------------------------------------------------
%	Invited Talks
%----------------------------------------------------------------------------------------

\section{Invited Talks and Invited Workshops}


\cventry{Jul 2021}{Sapienza University of Rome}{}{Gravity Theory Seminar}{}{}{}
\cventry{Apr 2021}{Universitat de les Illes Balears}{}{Seminar}{}{}{}
\cventry{Feb 2021}{Caltech}{}{Tapir Seminar}{}{}{}

\cventry{Dec 2020}{SISSA Trieste}{}{Gravity Seminar}{}{}{}

\cventry{Dec 2020}{TCNJ}{}{Physics Colloquium}{}{}{}


\cventry{Nov 2020}{Columbia University}{}{Theory Group Seminar}{}{}{}

\cventry{Oct 2020}{ICERM (Institute for Computational and Experimental Research in Mathematics), Brown University}{}{Mathematical and Computational Approaches for Solving the Source- Free Einstein Field Equations Workshop}{}{}{}

\cventry{Sep 2020}{ICERM (Institute for Computational and Experimental Research in Mathematics), Brown University}{}{Advances and Challenges in Computational Relativity Workshop}{}{}{}

\cventry{Aug - Sep 2020}{KITP (Kavli Institute of Theoretical Physics), UC Santa Barbara }{}{Probing Effective Theories of Gravity in Strong Fields and Cosmology Workshop}{}{}{}

\cventry{Aug 2020}{University of Mississippi}{}{Special seminar}{}{}{}

\cventry{June 2020}{Canadian Institute for Theoretical Astrophysics}{}{CITA seminar}{}{}{}

\cventry{July 2020}{Centro de Ciencias de Benasque}{}{New frontiers in Strong Gravity workshop}{\textit{Cancelled due to Covid-19 pandemic}}{}{}
\cventry{June 2020}{University of Rome}{}{Strong Gravity Beyond workshop}{\textit{Cancelled due to Covid-19 pandemic}}{}{} 


\cventry{Dec 2019}{NYU}{}{}{Guest lecture in general relativity course}{}
\cventry{Nov 2019}{University of Amsterdam}{}{Gravitational Wave Probes of Fundamental Physics workshop}{}{}{}
\cventry{Oct 2019}{NYU Center for Cosmology and Particle Physics}{}{Astro seminar}{}{}{}
\cventry{Dec 2018}{Cornell University}{}{Gravity Lunch Seminar}{}{}

\cventry{Nov 2018}{UT Austin}{}{Invited Seminar}{}{}
\cventry{Nov 2018}{Princeton University}{}{Princeton Gravity Initiative Lunch Seminar}{}{}
\cventry{Sep 2018}{Perimeter Institute}{}{Strong Gravity Seminar}{}{}

\cventry{Aug 2018}{Cal State Fullerton}{}{GWPAC High Performance Computing Workshop}{}{}

\cventry{July 2018}{Simons Summer Workshop}{}{Forefronts in Cosmology and Numerical General Relativity}{}{}

\cventry{June 2018}{Centro de Ciencias de Benasque}{}{Numerical Relativity beyond General Relativity workshop}{}{}

\cventry{April 2018}{Caltech}{}{Theoretical astrophysics seminar}{}{}

\cventry{Jan 2018}{Keck Institute for Space Sciences}{}{The Architecture of LISA Science Analysis}{}{}

\cventry{Dec 2017}{Caltech}{}{LIGO seminar}{}{}


%----------------------------------------------------------------------------------------
%	Honors and awards
%----------------------------------------------------------------------------------------

\section{Honors}


\cventry{June 2019}{Kip Thorne Prize}{for Excellence in Theoretical Physics}{Caltech}{}{}
\cventry{June 2018}{John Stager Stemple Memorial Prize}{for best performance on oral candidacy exam and research progress}{Caltech}{}{}{}
\cventry{Mar 2018}{American Physical Society DGRAV prize}{for best student talk at PCGM34}{Caltech}{}{}{}
\cventry{Oct 2017}{Oculus Prize}{Maestro team}{Hack Music LA}{}{}{}
\cventry{Oct 2017}{Amazon Prize}{Maestro team}{Hack Music LA}{}{}{}
\cventry{2014-2016}{Dominic Orr Graduate Fellowship}{full funding for first two years of research}{Caltech}{}{}{}
\cventry{July 2016}{Hartle Award}{for best talk in numerical relativity session}{GR21 conference}{}{}{}
\cventry{Nov 2015}{Theoretical Astrophysics in Southern California prize}{for best student talk}{Cal State Fullerton}{}{}{}
\cventry{June 2014}{Kusaka Memorial Prize in Physics}{for top graduating seniors in physics}{Princeton University}{}{}{}
\cventry{June 2013}{Allen G. Shenstone Prize in Physics}{for top juniors in physics}{Princeton University}{}{}{}


%----------------------------------------------------------------------------------------
%	Leadership and service
%----------------------------------------------------------------------------------------
\section{Service and Leadership}

\cventry{2019 - present}{Executive committee member}{Simulating eXtreme Spacetimes collaboration}{}{}{}
\cventry{2019 - present}{Student-Postdoc Advocate}{Simulating eXtreme Spacetimes collaboration}{}{}{}
\cventry{2019 - present}{Journal Referee}{APS Physical Review D, APS Physical Review Letters, Classical and Quantum Gravity}{}{}{}
\cventry{2017-2019}{Organizing committee member}{Caltech/JPL Association for Gravitational-Wave Research}{}{}{}
\cventry{2018}{Conference organizer}{Pacific Coast Gravity Meeting (PCGM) 34}{Caltech}{}{}
\cventry{2016-2017}{Graduate student organizer}{Theoretical astrophysics including relativity group}{Caltech}{}{}
\cventry{2015-2016}{Numerical relativity group discussion leader}{}{Caltech}{}{}
        

%----------------------------------------------------------------------------------------
%	Teaching
%----------------------------------------------------------------------------------------


\section{Teaching and mentorship}

\cventry{Summer 2021 - present}{\href{https://www.simonsfoundation.org/2020/11/30/simons-nsbp-scholars-program-2021/}{Simons-NSBP [National Society of Black Physicists] mentor}}{}{Mentoring UC Berkeley undergraduate student Lawrence Edmond in general relativity projects}{CCA}{}

\cventry{Summer 2020 - present}{\href{https://cunyastro.org/astrocom/}{AstroCom NYC mentor}}{}{Mentoring CUNY undergraduate students Destiny Howell and William Chakalis in projects in black hole and gravitational wave astrophysics. (Joint with Tom Callister)}{CCA / CUNY}{}

\cventry{2016-2017}{Teaching Assistant}{}{computational physics sequence (Ph20: Introduction to the Tools of Scientific Computing,  Ph21: Tools for Data Analysis, Ph 22: Tools for Numerical Methods)}{Caltech}{}

\cventry{Summer 2016}{Caltech SURF mentor}{}{}{Caltech}{}

\cventry{2012-2014}{Laboratory Teaching Assistant}{}{computer science sequence (COS 126: Introduction to Computer Science, COS 217: Introduction to Programming Systems, COS 226: Algorithms and Data Structure)}{Princeton University}{}
\medskip 

\cventry{}{I also maintain a \href{https://docs.google.com/document/d/1h4IOkTaq6E2bejXP07PSwa4KIIUjDjq-mtQlPtnAVOU/edit?usp=sharing}{guide} for undergraduates applying to physics graduate school}{}{}{}{}
%----------------------------------------------------------------------------------------
%	Outreach
%----------------------------------------------------------------------------------------



\section{Outreach}

\cvitem{}{I regularly participate in community science nights at local schools, guest lectures in high school and college courses, and astronomy outreach events including Astronomy on Tap. For an example of my public outreach talks to a general audience, please see a  \href{https://youtu.be/d0nHtoh6Mzk?t=224}{lecture on computational physics} I gave at Caltech. For an example of my outreach talks to K-12 students, please see one of the \href{https://www.dropbox.com/s/rdnketnevxjgmy9/2020-08-26\%20AaS\%20Okounkova.mov?dl=0}{Ask-a-Scientist} discussions I led at the Flatiron Institute. 
}






%----------------------------------------------------------------------------------------
%	References
%----------------------------------------------------------------------------------------

\section{References}

\setlength{\tabcolsep}{12pt}

\cvitem{}{\noindent \begin{tabular}{l l}
\href{http://astro.cornell.edu/members/saul-a-teukolsky.html}{\textbf{Prof. Saul Teukolsky}} & \href{http://pma.caltech.edu/content/mark-scheel}{\textbf{Research Prof. Mark Scheel}} \\
TAPIR, SXS Collaboration &  TAPIR, SXS Collaboration  \\
Caltech / Cornell &  Caltech \\
\small{\href{mailto:saul@astro.cornell.edu}{saul@astro.cornell.edu}} & \small{\href{mailto:scheel@tapir.caltech.edu}{scheel@tapir.caltech.edu}} \\ 
& \\
\href{https://duetosymmetry.com/}{\textbf{Asst. Professor Leo Stein}} &  \href{https://www.simonsfoundation.org/team/will-farr/}{\textbf{Prof. Will Farr}}  \\
University of Mississippi  & Flatiron CCA / Stony Brook University \\
\small{\href{mailto:leo.stein@gmail.com}{leo.stein@gmail.com}} & \small{\href{wfarr@flatironinstitute.org}{wfarr@flatironinstitute.org}}  \\
\end{tabular}}



%----------------------------------------------------------------------------------------

\end{document}



